\begin{abstractCN}
摘要应概括反映出毕业论文(设计)的内容、方法、成果和结论。摘要中一般不宜使用公式、图表,不标注引用文献编号。中文摘要以300左右字为宜、外文摘要以250个实词左右为宜。

关键词是供检索用的主题词条,应采用能覆盖毕业论文(设计)主要内容的通用技术词条(参照相应的技术术语标准),尽量从《汉语主题词表》中选用,未被词表收录的新学科、新技术中的重要术语和地区、人物、文献等名称,也可作为关键词标注。关键词一般为3~5个,按词条的外延层次排列(外延大的排在前面)。关键词应以与正文不同的字体字号编排在摘要下方。多个关键词之间用分号分隔。中英文关键词应一一对应。

    \keywordsCN 量子力学;算符次序;厄密算符;正则量子化;规范变换
\end{abstractCN}

\begin{abstractENG}
The abstract should summarize the content, methods, results, and conclusions of the thesis (project). Formulas, tables, and figures should generally not be used in the abstract, and reference numbers should not be cited. A Chinese abstract should be approximately 300 words, while a foreign abstract should be approximately 250 words.

Keywords are subject terms used for search queries. They should be general technical terms that cover the main content of the thesis (project) (referring to relevant technical terminology standards). Keywords should be selected from the Chinese Subject Headings List (HSL). Important terms in new disciplines and technologies not included in the HSL, as well as names of regions, individuals, and documents, may also be used as keywords. Keywords should generally be 3-5 in total, arranged according to the level of their extension (with the most extensive keywords appearing first). Keywords should be placed below the abstract in a different font size than the main text. Multiple keywords should be separated by semicolons. Chinese and English keywords should correspond one to one.



    \keywordsENG quantum mechanics;  operator ordering;  Hermitian operator;  canonical quantization; gauge transformation
\end{abstractENG}